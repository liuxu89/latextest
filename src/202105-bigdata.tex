\chapter{5.24 全国一体化大数据中心协同创新体系算力枢纽实施方案} % 2021

党的十八大以来,我国数字经济蓬勃发展,对构建现代化经济体系、实现高质量发展的支撑作用不断凸显。随着各行业数字化转型升级进度加快,特别是5G等新技术的快速普及应用,全社会数据总量爆发式增长,数据资源存储、计算和应用需求大幅提升,迫切需要推动数据中心合理布局、供需平衡、绿色集约和互联互通,构建数据中心、云计算、大数据一体化的新型算力网络体系,促进数据要素流通应用,实现数据中心绿色高质量发展。根据《关于加快构建全国一体化大数据中心协同创新体系的指导意见》(发改高技〔2020〕1922号)部署要求,为加快建设全国一体化大数据中心算力枢纽体系,制定本方案。

\section{总体要求}

{ \kaishu (一)指导思想。 }

以习近平新时代中国特色社会主义思想为指导,全面贯彻党的十九大和十九届二中、三中、四中、五中全会精神,深入落实习近平总书记关于建设全国一体化大数据中心的重要讲话精神,坚持新发展理念,坚持改革创新、先行先试,推动数据中心、云服务、数据流通与治理、数据应用、数据安全等统筹协调、一体设计,加快打造一批算力高质量供给、数据高效率流通的大数据发展高地。

{ \kaishu (二)基本原则。 }

{ \kaishu 加强统筹。} 加强数据中心统筹规划和规范管理,开展数据中心、网络、土地、用能、水、电等方面的政策协同,促进全国范围数据中心合理布局、有序发展,避免一哄而上、供需失衡。

{ \kaishu 绿色集约。} 推动数据中心绿色可持续发展,加快节能低碳技术的研发应用,提升能源利用效率,降低数据中心能耗。加大对基础设施资源的整合调度,推动老旧基础设施转型升级。

{ \kaishu 自主创新。} 以应用研究带动基础研究,加强对大数据关键软硬件产品的研发支持和大规模应用推广,尽快突破关键核心技术,提升大数据全产业链自主创新能力。

{ \kaishu 安全可靠。} 加强对基础网络、数据中心、云平台、数据和应用的一体化安全保障,提高大数据安全可靠水平。加强对个人隐私等敏感信息的保护,确保基础设施和数据的安全。

{ \kaishu (三)发展思路。 }

统筹围绕国家重大区域发展战略,根据能源结构、产业布局、市场发展、气候环境等,\cl{在京津冀、长三角、粤港澳大湾区、成渝,以及贵州、内蒙古、甘肃、宁夏等地布局建设全国一体化算力网络国家枢纽节点}(以下简称“国家枢纽节点”),发展数据中心集群,引导数据中心集约化、规模化、绿色化发展。国家枢纽节点之间进一步打通网络传输通道,加快实施“东数西算”工程,提升跨区域算力调度水平。同时,加强云算力服务、数据流通、数据应用、安全保障等方面的探索实践,发挥示范和带动作用。国家枢纽节点以外的地区,统筹省内数据中心规划布局,与国家枢纽节点加强衔接,参与国家和省之间算力级联调度,开展算力与算法、数据、应用资源的一体化协同创新。

\section{节点定位}

对于{ \bf 京津冀、长三角、粤港澳大湾区、成渝}等用户规模较大、应用需求强烈的节点,重点统筹好城市内部和周边区域的数据中心布局,实现大规模算力部署与土地、用能、水、电等资源的协调可持续,优化数据中心供给结构,扩展算力增长空间,满足重大区域发展战略实施需要。

对于{ \bf 贵州、内蒙古、甘肃、宁夏}等可再生能源丰富、气候适宜、数据中心绿色发展潜力较大的节点,重点提升算力服务品质和利用效率,充分发挥资源优势,夯实网络等基础保障,积极承接全国范围需后台加工、离线分析、存储备份等非实时算力需求,打造面向全国的非实时性算力保障基地。

对于国家枢纽节点以外的地区,重点推动面向本地区业务需求的数据中心建设,加强对数据中心绿色化、集约化管理,打造具有地方特色、服务本地、规模适度的算力服务。加强与邻近国家枢纽节点的网络联通。后续,根据发展需要,适时增加国家枢纽节点。

\section{数据中心布局}
按照绿色、集约原则,加强对数据中心的统筹规划布局,结合市场需求、能源供给、网络条件等实际,推动各行业领域的数据中心有序发展。原则上,将大型和超大型数据中心布局到可再生能源等资源相对丰富的区域,优化网络、能源等资源保障。在城市城区范围,为规模适中、具有极低时延要求的边缘数据中心留出发展空间,确保城市资源高效利用。

{ \kaishu (一)数据中心集群。 }

引导超大型、大型数据中心集聚发展,构建数据中心集群,推进大规模数据的“云端”分析处理,重点支持对海量规模数据的集中处理,支撑工业互联网、金融证券、灾害预警、远程医疗、视频通话、人工智能推理等抵近一线、高频实时交互型的业务需求,数据中心端到端单向网络时延原则上在20毫秒范围内。贵州、内蒙古、甘肃、宁夏节点内的数据中心集群,优先承接后台加工、离线分析、存储备份等非实时算力需求。

起步阶段,对于京津冀、长三角、粤港澳大湾区、成渝等跨区域的国家枢纽节点,原则上布局不超过2个集群。对于贵州、内蒙古、甘肃、宁夏等单一行政区域的国家枢纽节点,原则上布局1个集群。集群应注重集约化发展,明确数据中心建设规模、节能水平、上架率等准入标准,避免盲目投资建设。

{ \kaishu (二)城市内部数据中心。 }

在城市城区内部,加快对现有数据中心的改造升级,提升效能。支持发展高性能、边缘数据中心。鼓励城区内的数据中心作为算力“边缘”端,优先满足金融市场高频交易、虚拟现实/增强现实(VR/AR)、超高清视频、车联网、联网无人机、智慧电力、智能工厂、智能安防等实时性要求高的业务需求,数据中心端到端单向网络时延原则上在10毫秒范围内。

\section{国家枢纽节点重点任务}

{ \kaishu (一)加强绿色集约建设。 }

以数据中心集群布局等为抓手,加强绿色数据中心建设,强化节能降耗要求。推动数据中心采用高密度集成高效电子信息设备、新型机房精密空调、液冷、机柜模块化、余热回收利用等节能技术模式。在满足安全运维的前提下,鼓励选用动力电池梯级利用产品作为储能和备用电源装置。加快推动老旧基础设施转型升级。完善覆盖电能使用效率、算力使用效率、可再生能源利用率等指标在内的数据中心综合节能评价标准体系。

{ \kaishu (二)推动核心技术突破。 }

加大服务器芯片、操作系统、数据库、中间件、分布式计算与存储、数据流通模型等软硬件产品的规模化应用。支持和推广大数据基础架构、分布式数据操作系统、大数据分析等方面的平台级原创技术。组织科研院所、高校、企业、技术社区等力量协同研发和应用关键技术产品,提升大数据全产业链自主创新能力。

{ \kaishu (三)加快网络互联互通。 }

建设数据中心集群之间,以及集群和主要城市之间的高速数据传输网络,优化通信网络结构,扩展网络通信带宽,减少数据绕转时延。建立数据中心网络监测体系,推动数据中心与网络高效供给对接和协同发展。在国家枢纽节点内建立合理的网络结算机制,降低长途传输费用。围绕数据中心集群,稳妥有序推进国家新型互联网交换中心、国家互联网骨干直连点建设,促进互联网企业、云服务商、电信运营商等多方流量互联互通。

{ \kaishu (四)加强能源供给保障。 }

推动数据中心充分利用风能、太阳能、潮汐能、生物质能等可再生能源。支持数据中心集群配套可再生能源电站。扩大可再生能源市场化交易范围,鼓励数据中心企业参与可再生能源市场交易。支持数据中心采用大用户直供、拉专线、建设分布式光伏等方式提升可再生能源电力消费。保障数据中心用地和用水资源。

{ \kaishu (五)强化能耗监测管理。 }

建立健全数据中心能耗监测机制和技术体系。加强数据中心能耗指标统筹,从省区市层面对数据中心集群进行统一能耗指标调配,鼓励通过用能权交易配置能耗指标。探索开展跨省能耗和效益分担共享合作。鼓励数据中心在完成最低消纳责任权重的基础上,努力完成激励性消纳责任权目标。

{ \kaishu (六)提升算力服务水平。 }

支持政府部门和企事业单位整合内部算力资源,对集群和城区内部的数据中心进行一体化调度。支持在公有云、行业云等领域开展多云管理服务,加强多云之间、云和数据中心之间、云和网络之间的一体化资源调度。支持建设一体化准入集成验证环境,进一步打通跨行业、跨地区、跨层级的算力资源,构建算力服务资源池。

{ \kaishu (七)促进数据有序流通。 }

\cl{建设数据共享、数据开放、政企数据融合应用等数据流通共性设施平台,建立健全数据流通管理体制机制。试验多方安全计算、区块链、隐私计算、数据沙箱等技术模式,构建数据可信流通环境,提高数据流通效率。探索数据资源分级分类,研究制定相关规范标准。}

{ \kaishu (八)深化数据智能应用。 }

开展一体化城市数据大脑建设,为城市产业结构调整、经济运行监测、社会服务与治理、交通出行、生态环境等领域提供大数据支持。选择公共卫生、自然灾害、市场监管等突发应急场景,试验开展“数据靶场”建设,探索不同应急状态下的数据利用规则和协同机制。

{ \kaishu (九)确保网络数据安全。 }

完善海量数据汇聚融合的风险识别与防护技术、数据脱敏技术、数据安全合规性评估认证、数据加密保护机制及相关技术监测手段,同步规划、同步建设、同步使用安全技术措施,保障业务稳定和数据安全。加快推进全国互联网数据中心、云平台等数据安全技术监测手段建设,提升敏感数据泄露监测、数据异常流动分析等技术保障能力。

\section{保障措施}

{ \kaishu (一)加快推动落实。 }

各相关地区要高度重视,建立健全统筹协调和工作推进机制,明确责任部门,抓紧编制国家枢纽节点建设方案,统筹规划数据中心整合集约化建设,细化绿色发展目标,明确数据中心集群的布局、选址、规模、网络、用能,以及数据中心绿色节能等建设准入标准,提出深化各行业算力资源联通调度、促进数据资源流通应用等方面的政策改革举措和重大工程建议,报国家发展改革委、中央网信办、工业和信息化部和国家能源局。

{ \kaishu (二)加强政策支持。 }

发展改革委、中央网信办、工业和信息化部、国家能源局等部门组织相关评估机构和专家,加强对相关建设方案的评估指导。加大政策协同和支持力度,推动相关政策试点、工程试点优先在国家枢纽节点落地。重点围绕国家枢纽节点布局国家新型互联网交换中心、国家互联网骨干直连点等网络设施。对于符合条件且纳入国家枢纽节点数据中心集群范围的建设项目,积极协调安排能耗指标予以适当支持。依托国家政务信息化工程建设加强对政务大数据中心布局引导。对国家枢纽节点开展综合发展质量评估。

{ \kaishu (三)加强工程保障。 }

组织开展全国一体化大数据中心协同创新体系重大示范工程,在数据中心直连网络、一体化算力服务、数据流通和应用等领域开展试点示范,支持服务器芯片、云操作系统等关键软硬件产品规模化应用。支持开展“东数西算”示范工程,深化东西部算力协同。支持对大数据中心相关技术平台研制、资源接入调度、产业应用等共性技术和机制的集成验证
